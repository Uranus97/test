\documentclass[23pt,a4paper,two column]{article}
\setlength{\columnsep}{0.13\columnwidth}
%\setlength{\columnseprule}{0.05\columnwidth}
 
\usepackage{geometry}
\geometry{left=2.5cm,right=2.5cm,top=3.0cm,bottom=2.0cm}

\usepackage{fontspec}

\usepackage{xeCJK}

\usepackage{subfigure}

\usepackage{amssymb}

\usepackage{amsmath}

\usepackage{graphicx}

\usepackage{booktabs}

\usepackage{longtable}

\usepackage{tabularx}

\usepackage{wrapfig}

\usepackage{indentfirst}

\usepackage{float}								%超级好用!浮动排版!

\usepackage{flushend,cuted}
 
\usepackage{caption}
\captionsetup{font={scriptsize}}						%改变图名字体大小

\usepackage{fancyhdr}

\setlength{\parindent}{2em}

\linespread{1.3}

\begin{document}

\begin{strip}
	\begin{center}
	\huge{\bf{阿贝成像原理和空间滤波}}\\
	 \mbox{}\\ %插入一个空行
	\normalsize{物理学院\ 刘浚哲\ 1500011370}\\
	
	\end{center} 
	\mbox{}\\ 
	\rule{\textwidth}{0.13mm}\\ %画一条长横线
	\textbf {Abstract:}\\
	\hspace*{\parindent}本实验研究了傅立叶光学中空间频率、空间频谱和空间滤波等概念,了解了阿贝成像原理和透镜孔径对透镜成像分辨率的影响。\\
	 
	{\itshape Keywords:}\\
	\hspace*{\parindent}阿贝成像原理\;\;空间滤波\;\;光学傅立叶变换\;\;频谱面\;\;夫琅禾费衍射\;\;$\theta$调制\;\;卷积现象\\
	\rule{\textwidth}{0.13mm}\\
	
\end{strip}

\thispagestyle{plain}

\pagestyle{fancy}
\lhead{}
\rhead{\textbf{\thepage}}
\chead{\textit{J.Z.Liu\ / 普物实验\uppercase\expandafter{\romannumeral2}实验报告11(2017)5-19}}
\lfoot{}
\cfoot{}
\rfoot{}
\renewcommand{\headrulewidth}{1pt}
\renewcommand{\footrulewidth}{0}

\subsection*{1.实验现象记录与数据处理}
使用散斑原理找到透镜的后焦面, 记录透镜和其后焦面的坐标分别为$141.92cm$和$116.00 cm$,得到透镜的焦距为:
\begin{equation*}
	F=137.23-111.37= 25.86\;cm
\end{equation*}

\subsubsection*{1.1.一维光栅}
利用倒置的望远镜系统将激光束扩展成具有较大截面的平行光束, 用其照明一维光栅, 用光屏在频谱面接收, 在光屏上出现一排衍射光点。测量$\pm1,\pm2,\pm3$级点的坐标, 计算其空间频率$f_i=\Delta x/(2\lambda F)$,得到表1所示数据。表1中$x_-$表示负级条纹的坐标, $x_+$表示正级条纹的坐标。
\begin{table}[H]
\centering
\begin{tabular}{c|c|c|c}
	\hline
	\hline
	级别 &1&2&3\\
	\hline
	$x_-/mm$&-2.0&-4.0&-6.0\\
	
	$x_+/mm$&2.0&4.0&6.0\\
	
	$\Delta x/mm$&4.0&8.0&12.0\\
	
	$f_i/mm^{-1}$&12.2&24.4&36.7\\
	\hline
	\hline
\end{tabular}
\caption*{表1:测量一维光栅的空间频率}
\end{table}

可得一维光栅的基频为$12\;mm^{-1}$。

在频谱面上放置不同的光阑, 使得指定的衍射光斑通过。把滤波之后光栅所成的像放大后观察, 在不同的情况下观察到不同的现象。

在不设任何滤波器的情况下,衍射点全部通过,成像情况如图1:
\begin{figure}[H]
\centering
\includegraphics[scale=0.33]{/Users/macbookair/Desktop/Daily/实验/普物实验/普物实验(下)/空间滤波/1.jpg}
\caption*{图1:一维光栅衍射点全通成像图}
\end{figure}

可见在全通情况下,像面表现为边界清晰的纵向条纹。这其实是一维光栅通过透镜所成的像,与一维光栅透光的形状应该相同。经第二次透镜放大后,测量条纹的空间周期,大致为:
\begin{equation*}
d_1=2.8\;mm
\end{equation*}

\newpage

当用挡光片只挡住$0$级衍射点时,成像情况如图2:
\begin{figure}[H]
\centering
\includegraphics[scale=0.21]{/Users/macbookair/Desktop/Daily/实验/普物实验/普物实验(下)/空间滤波/2.jpg}
\caption*{图2:一维光栅只挡住0级衍射点成像图}
\end{figure}

此时可以观察到,相对于全通情形时,纵向条纹变细,且略暗。对比可以看到亮暗条纹发生反转。这是由于0级衍射点相当于直流成分,挡掉0级衍射点,相当于投射的波前函数
整体向下平移, 原来光强为0的点不再为0, 而成为光强极大点。此时测量条纹的空间周期,大致仍为:
\begin{equation*}
d_2=2.8\;mm
\end{equation*}
与全通情形一样。\\

当用挡光片挡住除$0$级的以外所有衍射点,只留下0级衍射点时,成像情况如图3:
\begin{figure}[H]
\centering
\includegraphics[scale=0.37]{/Users/macbookair/Desktop/Daily/实验/普物实验/普物实验(下)/空间滤波/3.jpg}
\caption*{图3:一维光栅只留下0级衍射点成像图}
\end{figure}

此时成像区域光强一片均匀,没有条纹。这是由于只通过0级成分,相当于只通过直流成分,不带来任何波动成分,因此光强不产生波动,为一常数。$d_3$不存在。\\

当用挡光片挡住$\pm1$级衍射点时,成像情况如图4:
\begin{figure}[H]
\centering
\includegraphics[scale=0.23]{/Users/macbookair/Desktop/Daily/实验/普物实验/普物实验(下)/空间滤波/4.jpg}
\caption*{图4:一维光栅只挡住$\pm1$级衍射点成像图}
\end{figure}

可以观察到,此时表现为纵向条纹,测量其空间周期,发现为:
\begin{equation*}
d_4=1.4\;mm
\end{equation*)
大致是全通情形周期的一半。

产生这种情形的原因在于:挡掉基频成分之后, 对像的贡献主要来自于常数项和二次谐频成分。而更高频率的光非常弱, 对于像的贡献非常小。所以在观察的时候会看到呈现出二倍与基频的空间频率。\\

当挡住除$0,\pm1$级以外的所有衍射点时,只露出$0,\pm1$级衍射点,成像情况如图5:

此时可以看见仍表现为纵向条纹,测量其空间周期,为:
\begin{equation*}
d_5=2.8\;mm
\end{equation*}
与全通情形一致。这是由于只通过$0,\pm1$级衍射点,相当于只通过了常数项和基频项,光强分布呈现出基频项的余弦型分布。


\newpage

\begin{figure}[H]
\centering
\includegraphics[scale=0.27]{/Users/macbookair/Desktop/Daily/实验/普物实验/普物实验(下)/空间滤波/5.jpg}
\caption*{图5:一维光栅只留下$0,\pm1$级衍射点成像图}
\end{figure}

\subsubsection*{1.2.二维光栅}

将二维光栅置于光路中, 在频谱面上可以观察到其二维衍射图样。放大衍射图样, 观察现象并记录其空间周期。

全通情形时,成像情况如图6:
\begin{figure}[H]
\centering
\includegraphics[scale=0.23]{/Users/macbookair/Desktop/Daily/实验/普物实验/普物实验(下)/空间滤波/6.jpg}
\caption*{图6:二维光栅全通情形成像图}
\end{figure}

可见像呈现为二维网格的形状,这是由于二维光栅直接在光屏上成的像, 和二维光栅有相同的结构,只不过经过了放大而已。测量其空间周期,得到:
\begin{equation*}
d_x=2.80\;mm\quad d_y=2.80\;mm
\end{equation*}

\newpage

当在频谱面只留下中心0级衍射点,挡住其余所有的光点时,没有条纹出现,光强为常数。如图7所示:
\begin{figure}[H]
\centering
\includegraphics[scale=0.37]{/Users/macbookair/Desktop/Daily/实验/普物实验/普物实验(下)/空间滤波/3.jpg}
\caption*{图7:二维光栅仅露出中心0级衍射点成像图}
\end{figure}

这是由于仅透过直流成分,而没有波动项,光强肯定为常数,不产生条纹。\\

当仅使中心的一横排光点透过时,成像情况如图8:
\begin{figure}[H]
\centering
\includegraphics[scale=0.21]{/Users/macbookair/Desktop/Daily/实验/普物实验/普物实验(下)/空间滤波/7.jpg}
\caption*{图8:二维光栅横狭缝成像图}
\end{figure}

此时却出现了纵向条纹,这是由于只通过横向的频谱, 纵向的空间周期性消失,因此表现出纵条纹。测量其空间周期,为:
\begin{equation*}
d_x=2.80\;mm
\end{equation*}

\newpage

当仅使中心的一纵排光点透过时,此时出现的是横向条纹,这是由于只通过纵向的频谱, 而横向的空间周期性消失了,因此表现出横条纹。测量其空间周期,为:
\begin{equation*}
d_y=2.80\;mm
\end{equation*}

当用斜狭缝挡住其他所有衍射点,仅通过一条斜线上的光点时,成像情况如图9:
\begin{figure}[H]
\centering
\includegraphics[scale=0.21]{/Users/macbookair/Desktop/Daily/实验/普物实验/普物实验(下)/空间滤波/8.jpg}
\caption*{图9:二维光栅斜狭缝成像图}
\end{figure}

此时出现的是斜狭缝,测量其垂直于斜线方向上的空间周期,为:
\begin{equation*}
d=1.97\;mm
\end{equation*}

大致为原来空间周期的$1/\sqrt{2}$倍,这是由于只通过了斜线方向上的频谱, 使这个方向的空间周期性消失, 产生了与其垂直方向的条纹。二维网格在45$^\circ$方向上光点的空间周期恰好是水平或者竖直方向的$\sqrt{2}$倍, 所以这个方向的空间频率也应该是水平或者竖直条纹的$1/\sqrt{2}$倍。

\subsubsection*{1.3.“光”字物屏}

将二维光栅换成“光”字光阑和正交光栅重叠的的物屏,调整物距使其通过透镜直接成像在远处。在频谱面上用光屏接收,观察到其频谱为分立点阵与连续谱的合成。在像面接收,其像边缘清晰,内部有点阵结构。

\newpage

在频谱面上放置$\phi=1mm$的圆形光阑进行滤波,使中心亮点通过,观察像的变化,如图10所示:
\begin{figure}[H]
\centering
\includegraphics[scale=0.43]{/Users/macbookair/Desktop/Daily/实验/普物实验/普物实验(下)/空间滤波/10.png}
\caption*{图10:“光”字物屏经$\phi=1mm$滤波成像图}
\end{figure}

与全通情形相比,此时字体边缘已经有点模糊了,且原来有的点阵结构已经消失。这是由于光阑滤掉了高频成分, 透过了低频成分, 所以边界会出现模糊的现象。透过的光斑是光栅的常数部分, 这部分是无法形成相位差的, 所以在像面上我们看不到"光"字内部有任何点阵结构。\\

在频谱面上放置$\phi=0.3mm$的圆形光阑进行滤波,使中心亮点通过,观察像的变化,如图11所示:
\begin{figure}[H]
\centering
\includegraphics[scale=0.33]{/Users/macbookair/Desktop/Daily/实验/普物实验/普物实验(下)/空间滤波/11.png}
\caption*{图11:“光”字物屏经$\phi=0.3mm$滤波成像图}
\end{figure}

与$\phi=1mm$滤波情形相比,此时字体边缘已经非常模糊了,且原来有的点阵结构也已经消失。这是由于滤波光阑更小使得透过的频率范围更加狭窄, 只有极低频率的光透过, 光强变化更加缓慢, 边缘更加模糊。

若想要使网格消失, 只需要不使基频及以上频率成分通过即可。基频成分为$f_0=12mm^{-1}$, 在频谱面上的坐标为:
\begin{equation*}
x' = \lambda F \cdot f_0=1.9\;mm
\end{equation*}
所以滤波器光阑半径必须小于1.9mm, 其孔径必须小于3.8mm。

使得字迹消失需要使低频成分通过, 滤掉高频成分。估计通过的频率上限为字迹宽度$W=0.5mm$的倒数, 则其在频谱面上的坐标为:
\begin{equation*}
x' = \frac{\lambda F}{W}=0.31\;mm
\end{equation*}
故滤波器的半径必须小于0.31mm, 其直径必须小于0.62mm。
	
在实际的实验过程中, 使用孔径为0.3mm的滤波器之后, 网格消失, 字迹已经非常模糊。预计使用更小孔径的时候字迹就会完全无法辨识。\\

移动频谱面上的光阑, 使得0级中心衍射点上方的$+1$级衍射点通过孔径为0.3mm的光阑, 观察到像的边界非常模糊,没有出现点阵结构,且亮度较原来更暗。

根据卷积定理,物的频谱为"光"字的连续谱和二维光栅离散谱的卷积,所以频谱面上每一个光点周围均包含了"光"字连续谱的所有低频信息。所以仅使$+1$级衍射点通过也能成一边界模糊的"光"字像。但由于二维光栅基频成分比0级成分弱,所以所成的像较暗。而又由于只通过了$+1$级衍射, 却并没有通过$-1$级衍射,所以光强不会有明显的纵向按基频变化的成分,内部不会有点阵结构出现。
	
\subsubsection*{1.4.“十”字物屏}

把十字物屏置于光路,在频谱面接收,可看到有横向,纵向两条两线和连续谱。在像面接收, 观察到边界清晰的十字像。如图12所示:\\
	
用光阑挡住频谱的中心成分,可以看到所成的像变得中心发暗而边缘发亮。这是因为内部主要是低频成分,边界主要是高频成分,光阑滤掉了低频成分的光,留下了高频成分的光,所以像会出现中心暗,边缘亮的现象。

\begin{figure}[H]
\centering
\includegraphics[scale=0.23]{/Users/macbookair/Desktop/Daily/实验/普物实验/普物实验(下)/空间滤波/9.jpg}
\caption*{图12:“十”字物屏成像图}
\end{figure}
	
\subsubsection*{1.5.$\theta$调制实验}

$\theta$调制是以不同取向的光栅调制物面图像上的不同部位,经空间滤波后,像面上各相应部位呈现不同的颜色。实验光路如图13所示:
\begin{figure}[H]
\centering
\includegraphics[scale=0.1]{/Users/macbookair/Desktop/Daily/实验/普物实验/普物实验(下)/空间滤波/12.jpeg}
\caption*{图13:$\theta$调制实验光路}
\end{figure}

图中S为光源, $L_1$, $L_2$为凸透镜, $P_1$为由薄膜光栅制成的样品作为物屏, $P_2$为S通过$L_1$成像的像面, 同时也是物屏的频谱面。在频谱面控制不同频率的光透过, 通过$L_2$成像于平面$P_3$上。
	
实验中作为物的样品由薄膜光栅组成。样品上的花、叶、盆等各部位光栅具有不同取向,相间角度为$60^\circ$。如图14所示:
\begin{figure}[H]
\centering
\includegraphics[scale=0.1]{/Users/macbookair/Desktop/Daily/实验/普物实验/普物实验(下)/空间滤波/13.jpeg}
\caption*{图14:$\theta$调制实验中的物面、频谱面与像面}
\end{figure}

实验中可以观察到在纸面上出现了三种颜色的衍射光点。用电烙铁在每种颜色出现的地方烫一个小洞, 可以看到$P_3$上出现了一个完整的红花绿叶黄盆的像。虽然只有一个小孔透光, 但是此时每个衍射点都含有像的完整信息, 自然就可以在$P_3$上完整成像。所成像如图15所示:
\begin{figure}[H]
\centering
\includegraphics[scale=0.23]{/Users/macbookair/Desktop/Daily/实验/普物实验/普物实验(下)/空间滤波/14.jpg}
\caption*{图15:$\theta$调制实验成像图}
\end{figure}

\subsubsection*{1.6.观察卷积现象}

用激光束分别照射20mm$^{-1}$和200mm$^{-1}$的两个正交光栅, 观察各自频谱。可以观察到20mm$^{-1}$光栅衍射所成的衍射点比较密集,而200mm$^{-1}$的光栅比较稀疏。
	
将二者重叠, 观察到频谱为20mm$^{-1}$和200mm$^{-1}$频谱之合成:即在200mm$^{-1}$光栅各分布稀疏的衍射点周围,还分布着一圈密集的20mm$^{-1}$光栅的衍射点。

转动20mm$^{-1}$的光栅,可以看到各密集的20mm$^{-1}$的频谱衍射点随之转动,但200mm$^{-1}$的光栅分布稀疏的衍射点并不随之转动;200mm$^{-1}$的光栅转动的时候,其对应的稀疏的频谱衍射点随之转动,但在这些点周围的20mm$^{-1}$的光栅分布密集的衍射点并不随之转动。可见这些衍射点的运动是独立的,并不随着另外一块光栅的运动而进行运动。

\newpage



\subsection*{2.收获与感想}

本次实验非常的有意思,从直观上展现了傅立叶光学的精髓。我们可以从频谱面上清晰地看到各个衍射点,并且遮住每一个衍射点都可以看到在像面上有相应的反馈。本次实验很好地将复杂、抽象的光学理论形象地演示了出来。从本次实验我们清晰地感受到光学的美,对于我们理解光学理论有非常大的帮助。

\rule{\columnwidth}{0.13mm}\\

\noindent\textbf{References:}\\
\scriptsize {[1] 吕斯骅,张朝晖,新编基础物理实验(第二版)(北京:高等教育出版社)第403-409页}\\

\begin{flushright}
\large\textbf{(指导老师:杨景)}
\end{flushright}

\raggedend

\end{document}